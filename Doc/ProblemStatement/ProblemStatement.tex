\documentclass[12pt]{article}

\usepackage{graphicx}
\usepackage{biblatex}
\usepackage{paralist}
\usepackage{listings}
\usepackage{hyperref}
\usepackage{soul}
\hypersetup{colorlinks=true,
    linkcolor=blue,
    citecolor=blue,
    filecolor=blue,
    urlcolor=blue,
    unicode=false}

\oddsidemargin 0mm
\evensidemargin 0mm
\textwidth 160mm
\textheight 200mm

\pagestyle {plain}
\pagenumbering{arabic}

\newcounter{stepnum}

\title{PocketSaver \\ Project Statement}
\author{Mevin Mathew , mathem1 , 400012057 \\ Shalmi Patel , patels19 , 400023762 \\ Diya Mathews , mathewsd , 400014156}
\date{September 25, 2017}

\begin {document}

\maketitle
\newpage
\tableofcontents

\newpage

\section{Problem Description}
According to a survey conducted by BMO, majority of Canadians spends about \$3,720 a year on impulse purchases. That comes up to around \$310 per month that could have been saved. Especially as a student living away from home, it is easy to buy a coffee here and a sandwich there without realizing how much we spend. We are implementing an application called PocketSaver, an easily accessible personal financing tool engineered to keep track of everyday expenses. This application will help the average person keep a record of their purchases, big or small, and promotes self-awareness of how much they spend. By visually seeing the amount spend, we can strive to shop smarter and save more. We are basing this project off another open sourced project called CoCoin. With our project, we strive to provide a solution to the problem of personal finance by keeping detailed records of everyday transactions. By keeping a detailed record these transactions, we promote better budgeting and money management. The key software development qualities that we are striving to meet are the usability, portability, correctness and reliability of the application. Our goal is to improve the usability of the application by creating a clean and intuitive user interface that allows the user to navigate the application with ease.\st {We will achieve portability by using Visual Studio, a multiplatform application development tool.} \textcolor{blue}{We will be using Visual studio, a muiltplatform application development tool.} With the open source project CoCoin and our own implementation, we strive to create an accurate and reliable personal financing tool for everyday use.

\section{Importance}
In our lives, we are always looking to save the most money that we can while maintaining a lifestyle that we are happy with. This app is beneficially to everyone, whether you are student, employee, etc. Many people keep track of their finances through planners and excel, for anything between every day purchases to house mortgages. This app will help organize a user’s expenditure and savings. This app will give a clear breakdown of the user’s finances graphically and mathematically. This will help the user plan how they spend their money and help them save. 

\section{Context of Problem}
\subsection{Stakeholders and Software Environment}
Our implementation of this personal finance application can be used by anyone who wants to develop or maintain better budgeting and money management skills. The application will be made by using a multiplatform development tool and therefore will be compatible with both \textcolor{blue}{Android} and \textcolor{blue}{iOS}. This allows the application to be easily accessible by everyone, especially the average student living on a budget. By reimplementing this project, we will ensure that PocketSaver has proper documentation, making it easier for future developers to contribute in the future.


\section{The Bibliography}{9}
CBC (2017). Canadians spend \$3,720 a year on impulse buys, survey finds. [online] Available at: http://www.cbc.ca/amp/1.1243111 [Accessed 25 Sep. 2017].

\end {document}
